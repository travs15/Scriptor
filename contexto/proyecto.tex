\chapter{Proyecto}
%<--\cite{LBol,AIsh,ATak}-->

\section{Misión}
	Somos una organización enfocada en la integridad y desarrollo de la creatividad de las personas que se sienten atraídas por las disciplinas literarias, generamos apoyo a las personas que se sienten atraídas a compartir sus pensamientos e ideas sin importar su edad, sexo, etc, brindamos una plataforma en la que el escritor amateur o profesional puede compartir escritos, recibir retroalimentación y generar expectativa en aquellos que los leen. también la difusión de sus ideas en diferentes plataformas, revistas y demás medios que sean posibles. 
	Trabajamos para construir una sociedad próspera, beneficiando el desarrollo de nuestra comunidad garantizando calidad y bienestar.


\section{Visión}
la visión de scriptor es la de brindar la posibilidad de difusión y conocimiento de talentos permitiendo a cualquier persona participar e incursionar en el mundo tanto de la escritura como la lectura. Procuramos brindar el mayor soporte a los escritores, proveyendo indicadores sobre su trabajo.

\section{Objetivos}

\subsection{Objetivos generales}
Incentivar las buenas relaciones humanas con el desarrallo de las artes literarias, propiciando el espacio adecuado para el intercambio de información en forma de escritos, impactando de manera positiva sobre la perspectiva del cliente.








\subsection{Objetivos específicos}
\begin{itemize}
	\item Mejora constante de la plataforma que se brinda para el seguimiento y apoyo de los usuarios.
	
	\item Generar un espacio de esparcimiento para la comunidad lectora y escritora incentivando el uso de scriptor.
	
	\item Propiciar semanalmente espacios de discusión orientados hacia temas de interés colectivo sobre diferentes escritos. 
	
	\item Brindar el servicio de una biblioteca virtual donde se almacenen los escritos de la comunidad, donde se puedan consultar libremente.
	
	\item Gestionar de forma óptima la comunicación, y realizar revisiones mensuales sobre las retroalimentaciones de nuestros usuarios.
	
	\item Gestionar encuentros de la comunidad se Scriptor fomentando su uso,obteniendo retroalimentacion mas directa del usuario.
	
\end{itemize}

